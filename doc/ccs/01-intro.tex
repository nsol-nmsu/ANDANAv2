\section{Introduction} \label{sec:introduction}
Usage of the Internet has undergone a tremendous trasnformation since its inception in the 1970s. Content distribution, as opposed to point-to-point communication, has become the leading type of traffic traversing today's valuable network resources; Netflix, for example, accounted for nearly 30\% of all downstream traffic in 2012 \cite{Netflix}. The number and popularity of such information-centric services are only expected to increase in the future with the growing presence of data-intensive consumer applications and devices (e.g., media streaming applications and mobile devices), leading to added pressure on network resources and a subsequent increase in network congestion and wasted bandwidth. 

Named-data networking (NDN) \cite{ndn-techreport} is an emerging network architecture capable of supporting information-centric traffic. Two primary characteristics of the NDN architecture are that content names, rather than hosts or locations, are addressible and routable through the network, and all generated content corresponding to some name must be signed by its original producer. The latter property decouples content confidentiality, integrity, and authenticity from content and the manner in which it is delivered to consumers (e.g., instead of using secure tunnels akin to SSH/TLS, content can be encrypted before sent to a consumer). These fundamental design decisions enable content to be cached in network-layer resources throughout the network, thus promoting reduced network congestion and wasted bandwidth when popular content is requested. 

The NDN design has strong implications on content security and consumer privacy. 

TODO: introduce privacy notions from ANDANA paper, introduce ANDANA application (below), and then motivate ANDANAv2

The primary motivation for a new desgin of {\sf AND\=aNA} is to attain the same anonymity and privacy guarantees as {\sf AND\=aNA} with \emph{better} performance. The original design targeted a single use case in which performance, especially in the bidirectional setting, was not a primary concern. Indeed, there was both an asymmetric and symmetric (session-based) variant of {\sf AND\=aNA}, and while the latter enjoyed better speedups over the former it suffered the fatal flaw of not ensuring packet unlinkability. It is generally the case that unlinkability is merely sufficient for anonymity, rather than also being a necessary condition for anonymity. However, in the case of {\sf AND\=aNA}, packet linkability can lead to consumer and producer linkability, which immediately violates anonymity. For example, it is not difficult to hypothesize an adversary that eavesdrops on incoming and outgoing interests for a particular anonymous router, and who by doing so is able to determine that the incoming and outgoing session IDs are linked. In fact, a modified type of this kind of adversary was explicitly studied in the context of Tor by Murdoch and Danezis in \cite{tor-traffic-analysis}. In their work, the goal of the adversary in their ``linkability attack'' was to determine whether two separate data streams being served by two corrupted servers were initiated by the same consumer, and we suspect that such analysis could be augmented to work for {\sf AND\=aNA}. Specifically, repeating a packet linkability attack at each anonymous router in a circuit may therefore eventually lead to linkability between the producer and consumer. The use of application and environment contextual information has also been formally studied in \cite{attacking-unlinkability}, in which side channel and environment information (e.g., the deterministic behavior of an anonymous router always forwarding a packet upstead after unwrapping an interest received from some downstream router) is used to quantify the \emph{degree of unlinkability}. Furthermore, we remark that regardless of how such linkability information is acquired, it has been shown that it can lead to reduce consumer and producer anonymity beyond what is possible with general traffic analysis \cite{linkability-attacks}. 

As most of the literature focuses on mix-based anonymizing services like Tor, which inspired the original design of {\sf AND\=aNA}, it is clear that any form of linkability should be avoided in order to maintain consumer and producer anonymity. Therefore, the formal goal for {\sf AND\=aNA-v2} is to attain the same anonymity and privacy guarantees as the asymmetric variant of {\sf AND\=aNA}, which does not suffer from packet linkability issues, while supporting high-throughput and low-latency traffic between two parties. 

